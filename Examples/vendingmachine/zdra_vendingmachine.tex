\documentclass{article}
\usepackage{zmathlang}

\begin{document}

\dratheory{T1}{0.5}{
\begin{zed}
price:\nat
\end{zed}

\draschema{SS1}{
\begin{schema}{VMSTATE}
stock, takings: \nat
\end{schema}}

\draschema{CS0}{
\begin{schema}{VM\_operation}
\Delta VMSTATE \\
cash\_tendered?, cash\_refunded!: \nat \\
bars\_delivered! : \nat
\end{schema}}

\uses{CS0}{SS1}

\draschema{PRE1}{
\begin{schema}{exact\_cash}
cash\_tendered?: \nat
\where
cash\_tendered? = price
\end{schema}}

\draschema{PRE2}{
\begin{schema}{insufficient\_cash}
cash\_tendered? : \nat
\where
cash\_tendered? < price
\end{schema}}

\draschema{PRE3}{
\begin{schema}{some\_stock}
stock: \nat
\where
stock > 0
\end{schema}}

\draschema{CS1}{
\begin{schema}{VM\_sale}
VM\_operation
\where
\draline{PO1}{stock' = stock -1 \\
bars\_delivered! = 1 \\
cash\_refunded! = cash\_tendered? - price \\
takings' = takings + price}
\end{schema}}


\uses{CS1}{CS0}
\requires{CS1}{PO1}

\draschema{CS2}{
\begin{schema}{VM\_nosale}
VM\_operation
\where
\draline{PO2}{stock' = stock \\
bars\_delivered! = 0 \\
cash\_refunded! = cash\_tendered?\\
takings' = takings}
\end{schema}}

\uses{CS2}{CS0}
\requires{CS2}{PO2}

\draschema{TS1}{
\begin{zed}
VM1 \defs exact\_cash \land some\_stock \land VM\_sale
\end{zed}}

\uses{TS1}{PRE1}
\uses{TS1}{PRE3}
\uses{TS1}{CS1}

\draschema{TS2}{
\begin{zed}
VM2 \defs insufficient\_cash \land VM\_nosale
\end{zed}}

\uses{TS2}{PRE2}
\uses{TS2}{CS2}

\draschema{TS3}{
\begin{zed}
VM3 \defs VM1 \lor VM2
\end{zed}}
\totalises{TS3}{TS1}
\totalises{TS3}{TS2}
}


\begin{tabular}{|l|}
\hline
\textbf{Assigned rhetorical roles}\\
\hline
(T1, hasRhetoricalRole, theory) \\
(SS1, hasRhetoricalRole, stateschema) \\
(CS0, hasRhetoricalRole, changeschema) \\
(CS1, hasRhetoricalRole, changeschema) \\
(CS2, hasRhetoricalRole, changeschema) \\
(PRE1, hasRhetoricalRole, precondition) \\
(PRE2, hasRhetoricalRole, precondition) \\
(PRE3, hasRhetoricalRole, precondition) \\
(PO1, hasRhetoricalRole, postcondition) \\
(PO2, hasRhetoricalRole, postcondition) \\
(TS1, hasRhetoricalRole, totalise) \\
(TS2, hasRhetoricalRole, totalise) \\
(TS3, hasRhetoricalRole, totalise) \\
\hline
\hline
\textbf{Relations}\\
\hline
(CS0, uses, SS1)\\
(CS1, uses, CS0)\\
(CS2, uses, CS)\\
(PRE1, allows, PO1)\\
(PRE3, allows, PO1)\\
(PRE2, allows, PO2)\\
TS1, totalises, PRE1 \\
TS2, totalises, PRE2 \\
TS1, totalises, PRE3 \\
TS3, totalises, TS1 \\
TS3, totalises, TS2 \\
\hline
\end{tabular}

\end{document}