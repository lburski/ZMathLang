\documentclass{article}
\usepackage{zmathlang}
\pagenumbering{gobble}

\begin{document}

\begin{enumerate}
\item The mode-control panel contains four buttons for selecting modes and three displays for
dialing in or displaying values. The system supports the following four modes:

\begin{itemize}
\item attitude control wheel steering (att\_cws)
\item flight path angle selected (fpa\_sel)
\item altitude engage (alt\_eng)
\item calibrated air speed (cas\_eng)
\end{itemize}

\begin{zed}
events ::= press\_att\_cws |  press\_cas\_eng | press\_alt\_eng | \\
press\_fpa\_sel
\end{zed}

Only one of the first three modes can be engaged at any time. However, the cas\_eng mode
can be engaged at the same time as any of the other modes. The pilot engages a mode by pressing
the corresponding button on the panel. One of the three modes, att\_cws, fpa\_sel, or alz\_eng,
should be engaged at all times. Engaging any of the first three modes will automatically cause the
other two to be disengaged since only one of these three modes can be engaged at a time.

\begin{zed}
mode\_status ::= off | engaged
\end{zed}


\begin{schema}{off\_eng}
mode: mode\_status
\where
mode = off \lor mode = engaged
\end{schema}

\begin{schema}{AutoPilot}
att\_cws: mode\_status \\
fpa\_sel: mode\_status \\
alt\_eng: mode\_status \\
cas\_eng: mode\_status
\end{schema}

\begin{schema}{att\_cwsDo}
\Delta AutoPilot 
\where
att\_cws = off \\
att\_cws' = engaged \\
fpa\_sel' = off \\
alt\_eng' = off \\
cas\_eng' = off \lor engaged \\
\end{schema}

\item There are three displays on the panel: and altitude [ALT], flight path angle [FPA], and
calibrated air speed [CAS]. The displays usually show the current values for the altitude, flight
path angle, and air speed of the aircraft. However, the pilot can enter a new value into a display by dialing in the value using the knob next to the display. This is the target or "pre-selected" value that the pilot wishes the aircraft to attain. For example, if the pilot wishes to climb to 25,000 feet, he will dial 25,000 into the altitude display window and then press the alz\_eng button to engage the altitude mode. Once the target value is achieved or the mode is disengaged, the display reverts to showing the "current" value.

\item If the pilot dials in an altitude that is more than 1,200 feet above the current altitude and
then presses the alz\_eng button, the altitude mode will not directly engage. Instead, the altitude
engage mode will change to "armed" and the flight-path angle select mode is engaged. The pilot
must then dial in a flight-path angle for the flight-control system to follow until the aircraft attains the desired altitude. The flight-path angle select mode will remain engaged until the aircraft is within 1,200 feet of the desired altitude, then the altitude engage mode is automatically engaged.

\item The calibrated air speed and the flight-path angle values need not be pre-selected before the
corresponding modes are engaged--the current values displayed will be used. The pilot can dial-in
a different target value after the mode is engaged. However, the altitude must be pre-selected
before the altitude engage button is pressed. Otherwise, the command is ignored.

\item The calibrated air speed and flight-path angle buttons toggle on and off every time they are
pressed. For example, if the calibrated air speed button is pressed while the system is already in
calibrated air speed mode that mode will be disengaged. However, if the attitude control wheel
steering button is pressed while the attitude control wheel steering mode is already engaged, the
button is ignored. Likewise, pressing the altitude engage button while the system is already in
altitude engage mode has no effect.

Because of space limitations, only the mode-control panel interface itself will be modeled in this
example. The specification will only include a simple set of commands the pilot can enter plus the functionality neededto support modes switching and displays.The actual commands that would
be transmitted to the flight-control computerto maintain modes,etc., are not modeled.

\end{enumerate}

\end{document}